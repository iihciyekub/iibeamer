%-------------------------------------------------------------------------
% 字体设定相关
%-------------------------------------------------------------------------
\usepackage{ctex}																		% 调用CJK(中日韩)字体宏包

%--英文
%\setmainfont{latinmodernmath.otf}														% 全局西文使用指定字体(author:对beamer无效)
\setsansfont[Path = Theme/Fonts/,AutoFakeBold,AutoFakeSlant]{latinmodernmath.otf}				% 全局西文使用指定字体(author:对beamer有效)
\newfontfamily\latmath[Path = Theme/Fonts/,AutoFakeBold,AutoFakeSlant]{latinmodernmath.otf}	% 加载目录中指定数学字母字体
\newfontfamily\euler[Path = Theme/Fonts/,AutoFakeBold,AutoFakeSlant]{euler.otf}				% 加载目录中指定英文字体

%--中文
%\setCJKmainfont{KaiTi.ttf}																% 全局中文使用指定字体 (author:对beamer无效)
\setCJKsansfont[Path = Theme/Fonts/,AutoFakeBold,AutoFakeSlant]{KaiTi.ttf}					% 全局中文使用指定字体 (author:对beamer有效)
%\setCJKfamilyfont{kt}[Path = fonts/]{KaiTi.ttf}										% ::{自定义的CJKfamily名称}{字体名称} (author:命令太长弃用)
%\newcommand{\KT}{\CJKfamily{kt}}														% ::{自定义指令名}{\CJKfamily{自定义的CJKfamily名称}}(author:命令太长弃用)
\let\kaishu\relax 																		% 清除旧定义
\newCJKfontfamily\FZLBFW[Path = Theme/Fonts/,AutoFakeBold,AutoFakeSlant]{FZLBFW.TTF}			% 加载指定目录的中文字体,并设定粗体,斜体
\newCJKfontfamily\FZLBJW[Path = Theme/Fonts/,AutoFakeBold,AutoFakeSlant]{FZLBJW.TTF} 			% 加载指定目录的中文字体,并设定粗体,斜体


%--特殊字体,比如符号,小图标
\RequirePackage{Theme/Fonts/fontawesome}
\newfontfamily{\FA}[Path = Theme/Fonts/,AutoFakeBold,AutoFakeSlant]{FontAwesome.otf}

\CTEXoptions[today=old]																	% 使用旧的时间格式

%---------------
%tikz(形状图形) ,pgf(画函数图) 相关库		%
%---------------
\RequirePackage{tikz}
\usetikzlibrary{bending,arrows,shapes,graphs,spy,datavisualization}
\usetikzlibrary{backgrounds,positioning,fit,petri}
\usetikzlibrary{decorations.fractals,decorations.pathmorphing}							% 分形图宏包
\usetikzlibrary{math}																	% 用于\tikzmath 命令
\usetikzlibrary{calc}																	% 用于计算($()+()$) 之类的命令
\usepackage{pgfplots}																	% PGF宏包
\pgfplotsset{compat=1.16}
\usepackage{pgfmath}																	% 用于 \pgfmathparse 的命令


%---------------
%动画宏包		%
%---------------
\usepackage{graphicx}																	
\usepackage{animate}


%---------------
%math相关宏包		%
%---------------
\usepackage{extarrows}
\usepackage{ifthen}																% 引入判断语句宏包
\usepackage{wasysym}															% 数学符号宏包
\usefonttheme[onlymath]{serif}													% 设置数学公式字体



%---------------
%其它宏名		%
%---------------
\usepackage{hyperref}
%\hypersetup{pdfpagemode=FullScreen}												% 默认开启PDF 全屏
% 二维码宏包
\usepackage{qrcode}


